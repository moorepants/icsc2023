\documentclass{icsc}

\begin{document}

\begin{center}
  \fontsize{14}{20}{\bf Instructions for Preparing an Abstract for \\[3pt]
                        the International Cycling Safety Conference 2017}
\end{center}

%%%%%%%%%%%%%%%% authors %%%%%%%%%%%%%%%
\begin{center}
  \normalsize{\bf{A. B. Authorone$^{*}$,
                  C. Authortwo$^\#$,
                  D. E Authorthree$^\dag$}}
\end{center}

\begin{center}
  \begin{tabular}{c}
    $^*$ Faculty of Mechanical Engineering\\
    University of Technology\\
    Address, Postcode City, Country\\
    e-mail: email1@address\\
  \end{tabular}
  \begin{tabular}{c}
    $^\#$ Institute for Mechatronics\\
     University of Technology\\
    Address, Postcode City, Country\\
    e-mail: email2@address\\
  \end{tabular} \\ \vspace{2.5ex}
  \begin{tabular}{c}
    $^\dag$ Laboratory for Engineering Mechanics\\
     Delft University of Technology\\
     Mekelweg 2, NL-2628~CD~~Delft, The Netherlands\\
     e-mail: D.E.Authorthree@tudelft.nl\\
  \end{tabular}
\end{center}

\begin{keywords}
guidelines for authors,
template,
final symposium paper,
formatting instructions.
\end{keywords}

\section{INTRODUCTION}
%
The first page begins with the title of the paper, the authors, affiliations,
and keywords as it is shown in this document. The title of the paper should be
typed in 14 pt bold font and be centered. Authors' names should be typed in
bold font. Affiliations should be typed in a general formatting style, centered
within a corresponding bounding box. The authors should be followed by a list
of keywords after the header ``Keywords:'' typed in bold 11 pt font. The
keywords should be separated by commas. A dot should be place after the last
keyword. Up to five keywords are allowed. Title, affiliations, and keywords
have to fit on the first page. The introduction paragraph may start on the
first page provided there is enough space.

All text should be written in 11 pt Liberation Serif font. General text should
be justified and divided into logical paragraphs separated by 11 pt spaces.
Single line spaces are required. All section headers except the references
header should be numbered with Arabic numbers without a trailing dot. Top level
headers should be typed in capital characters. Subsections should be numbered
with multilevel numbering. A maximum of two sublevels is allowed. Subheaders
should be typed in normal word case. Font size of 11 pt should be used if not
specified otherwise.

% TODO : The Liberation Serif font is only available if XeTeX or LuaTex is
% used. This was the default font in the Libreoffice template. For now, just
% use the default font that PDFLaTeX can handle.

\section{GENERAL INSTRUCTIONS}
%
The abstract must be written in English. It must contain the name, address and
e-mail address of each author. The abstract should be no longer than three
pages including references. All page margins should be 1''. The paper used
should be letter size (8.5'' $\times$ 11.0''). It is suggested to use styles
for formatting and automatic reference, figure numbering to avoid editorial
errors.  To avoid compatibility problems it is advised to use only upper or
lower case Latin alphabet, numbers and the underscore character in the file
name.

\subsection{Equations}
%
Equations should be numbered continuously according to the format shown in
Equation~(\ref{eq:equ1}):
\begin{equation} \label{eq:equ1}
  e^{i\pi} + 1 = 0,
\end{equation}
where the unknown symbols are explained after the equation.

\subsection{Figures and tables}
%
\begin{table}[h!]
  \begin{center}
    \caption{Example of a table with a short caption.} \label{tab:tab1}
    \begin{tabular}{|c|ccc|}
      \hline
      &  $x$  &  $y$  &  $z$ \\
      \hline
      $x'$  &  $\alpha_1$ & $\beta_1$ & $\gamma_1$ \\
      $y'$  &  $\alpha_2$ & $\beta_2$ & $\gamma_2$ \\
      $z'$  &  $\alpha_3$ & $\beta_3$ & $\gamma_3$ \\
      \hline
    \end{tabular}
  \end{center}
\end{table}
All figures should be clearly readable and relevant to the presented text. Use
of at least 300\,dpi resolution for pictures and 600\,dpi for line art is
required, 1\,px wide lines in figures should be avoided as they may become
invisible in print. There is no limit on the amount of figures as long as they
do not dominate the text and the total length of the paper is within the
specified limits. Both figures and tables should be centred on the page.

\begin{figure}[h!]
\begin{center}
  \includegraphics[width=55mm]{figure1}
  \caption{An example of a figure caption. Use 10~pt Times New Roman.
           For long captions, use a text width of 13~cm.
           Use the same style for the tables.} \label{fig:fig1}
\end{center}
\end{figure}
Figures, graphs and tables must be included in the same style as shown for
Figure~\ref{fig:fig1} and Table~\ref{tab:tab1}.


\subsection{References}
%
Bibliographical citations should be written in the order in which they are
cited, see the References section below, where Reference~\cite{Pac02}
exemplifies the case of a textbook, while Reference~\cite{Ber07} is an article
in conference proceedings and Reference~\cite{Sha71} is an article in a
journal. Use can be made of a pre-existing bibtex style which uses a similar
style.

\section{PROCEEDINGS}
%
The final version of your extended abstract will be part of the proceedings.
These proceedings will be made available to the conference participants and the
public at the start of the conference via a citeable online repository.

\section{CONCLUSIONS}
%
We very much look forward to welcoming you to Davis! Best wishes and the
warmest regards from the Organizing Committee of International Cycling Safety
Conference 2017.

\begin{thebibliography}{6}
% \begin{thebibliography}{66} % use this for more than 9 references
  \bibitem{Pac02} H.~B.~Pacejka, \textit{Tyre and Vehicle Dynamics},
    Butterworth and Heinemann, Oxford, 2002.
  \bibitem{Ber07} E.~Bertolazzi, F.~Biral, M.~Da~Lio and V.~Cossalter, ``The
    influence of rider's upper body motions on motorcycle minimum time
    maneuvering'', in C.~L.~Bottasso, P.~Masarati and L.~Trainelli (eds),
    \textit{Proceedings, Multibody Dynamics 2007, ECCOMAS Thematic Conference},
    Milano, Italy, 25--28 June 2007, Politecnico di Milano, Milano, 2007,
    15~pp.
  \bibitem{Sha71} R.~S.~Sharp, ``The stability and control of motorcycles'',
    \textit{Proceedings of the IMechE, Part C, Journal of Mechanical
    Engineering Science} \textbf{13} (1971), pp.~316--329.
\end{thebibliography}

\end{document}

